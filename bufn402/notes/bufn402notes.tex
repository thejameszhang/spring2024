\documentclass[12pt]{scrartcl}
\usepackage[sexy]{james}
\usepackage[noend]{algpseudocode}
\setlength{\marginparwidth}{2cm}
\usepackage{answers}
\usepackage{array}
\usepackage{tikz}
\newenvironment{allintypewriter}{\ttfamily}{\par}
\usepackage{listings}
\usepackage{xcolor}
\usetikzlibrary{arrows.meta}
\usepackage{color}
\usepackage{mathtools}
\newcommand{\U}{\mathcal{U}}
\newcommand{\E}{\mathbb{E}}
\usetikzlibrary{arrows}
\Newassociation{hint}{hintitem}{all-hints}
\renewcommand{\solutionextension}{out}
\renewenvironment{hintitem}[1]{\item[\bfseries #1.]}{}
\renewcommand{\O}{\mathcal{O}}
\declaretheorem[style=thmbluebox,name={Chinese Remainder Theorem}]{CRT}
\renewcommand{\theCRT}{\Alph{CRT}}
\setlength\parindent{0pt}
\usepackage{sansmath}
\usepackage{pgfplots}

\usetikzlibrary{automata}
\usetikzlibrary{positioning}  %                 ...positioning nodes
\usetikzlibrary{arrows}       %                 ...customizing arrows
\newcommand{\eqdef}{=\vcentcolon}
\newcommand{\tr}{{\rm tr\ }}
\newcommand{\im}{{\rm Im\ }}
\newcommand{\spann}{{\rm span\ }}
\newcommand{\Col}{{\rm Col\ }}
\newcommand{\Row}{{\rm Row\ }}
\newcommand{\dint}{\displaystyle\int}
\newcommand{\dt}{\ {\rm d }t}
\newcommand{\PP}{\mathbb{P}}
\newcommand{\Var}{\text{Var}}
\newcommand{\horizontal}{\par\noindent\rule{\textwidth}{0.4pt}}
\usepackage[top=3cm,left=3cm,right=3cm,bottom=3cm]{geometry}
\newcommand{\mref}[3][red]{\hypersetup{linkcolor=#1}\cref{#2}{#3}\hypersetup{linkcolor=blue}}%<<<changed

\tikzset{node distance=4.5cm, % Minimum distance between two nodes. Change if necessary.
         every state/.style={ % Sets the properties for each state
           semithick,
           fill=cyan!40},
         initial text={},     % No label on start arrow
         double distance=4pt, % Adjust appearance of accept states
         every edge/.style={  % Sets the properties for each transition
         draw,
           ->,>=stealth',     % Makes edges directed with bold arrowheads
           auto,
           semithick}}


% Start of document.
\newcommand{\sep}{\hspace*{.5em}}

\pgfplotsset{compat=1.18}
\begin{document}
\title{BUFN402: Portfolio Management}
\author{James Zhang\thanks{Email: \mailto{jzhang72@terpmail.umd.edu}}}
\date{\today}

\definecolor{dkgreen}{rgb}{0,0.6,0}
\definecolor{gray}{rgb}{0.5,0.5,0.5}
\definecolor{mauve}{rgb}{0.58,0,0.82}

\lstset{frame=tb,
  language=Java,
  aboveskip=3mm,
  belowskip=3mm,
  showstringspaces=false,
  columns=flexible,
  basicstyle={\small\ttfamily},
  numbers=left,
  numberstyle=\tiny\color{gray},
  keywordstyle=\color{blue},
  commentstyle=\color{dkgreen},
  stringstyle=\color{mauve},
  breaklines=true,
  breakatwhitespace=true,
  tabsize=3
}

\maketitle
    These are my notes for UMD's BUFN402: Portfolio Management, 
    which is an elective (``live-\TeX``-ed). This course is taught by Professor Seokwoo Lee. 
\tableofcontents
\newpage

\section{Probability Theory}

See Professor Lee's comprehensive slideshow on the probability theory concepts that 
we need to know.

\section{Portfolio Theory}

\subsection{Review of Two Asset Case}

Suppose we have two assets denoted as random variables $R_1, R_2$ and $E(R_1) = \mu_1, E(R_2) = \mu_2$. 
We also have $SD(R_1) = \sigma_1, SD(R_2)= \sigma_2$. Then we have the portfolio $R_p$ as
\[R_p = w_1R_1 + w_2R_2 = w_1R_1 + (1 - w_1)R_2\]
Now constucting a mean variance optimization, 
\[\E(R_p) = w_1\mu_1 + w_2\mu_2\]
\[\sigma_p^2 = w_1^2\sigma_1^2 + (1-w_1)^2\sigma_2^2 + 2w_1(1-w_1)\rho_{12}\sigma_1\sigma_2\]
\[\sigma_p = \sqrt{\sigma_p^2}\]
We wish to maximize $\max(\E(R_p) - \text{Var}(R_p))$ which we can doing by taking the 
derivative with respect to $w_1$ and setting this to zero. 
\[w_1\mu_1 + (1-w_1)\mu_2 - w_1^2\sigma_1^2 + (1-w_1)^2\sigma_2^2 + 2w_1(1-w_1)\rho_{12}\sigma_1\sigma_2\]
\[w_1\mu_1 + \mu_2-w_1\mu_2 - w_1^2\sigma_1^2 + (1-2w_1 + w_1^2)\sigma_2^2 + 2w_1(1-w_1)\rho_{12}\sigma_1\sigma_2\]
\[w_1\mu_1 + \mu_2-w_1\mu_2 - w_1^2\sigma_1^2 + (1-2w_1 + w_1^2)\sigma_2^2 + (2w_1-2w_1^2)\rho_{12}\sigma_1\sigma_2\]
Taking partial derivative we get
\[\mu_1 - \mu_2 - 2\sigma_1^2w_1 - 2\sigma_2^2 + 2w_1\sigma_2^2 + (2-4w_1)\rho_{12}\sigma_1\sigma_2 = 0\]
\[\cdots\]

\begin{example}
  We want to maximize variance of two assets such that $X \sim N(\mu_x, 1), Y \sim N(\mu_y, 1), \rho > 0$. 
  Maximize the variance given that $w_1^2 + w_1^2 = 1$. Note that maximizing this variance 
  is the same thing as minimizing the negative value of this function.
  \[\Var(R_p) = w_1^2 + w_2^2 + 2w_1w_2\rho\]
  \[-\Var(R_p) = -w_1^2 -  w_2^2 - 2w_1w_2\rho\]
  Rearranging the constraint yields 
  \[1 - w_1^2 - w_1^2\]
  \[\mathcal{L}(w_1, w_2, \lambda) = -w_1^2 -w_2^2 -2w_1w_2\rho + \lambda(1 - w_1^2 - w_2^2)\]
  The first order conditions are therefore
  \[\frac{\partial \mathcal{L}}{\partial w_1} = -2w_1 - 2w_2\rho -2\lambda w_1 = 0 \implies w_1 + w_2\rho + \lambda w_1 = 0\] 
  \[\frac{\partial \mathcal{L}}{\partial w_2} = -2w_2  - 2w_1\rho - 2\lambda w_2 = 0 \implies w_2 + w_1\rho + \lambda w_2 = 0\]
  We want to solve for $w_1, w_2$ in terms of $\lambda$. Recall that the constraint enforces 
  $w_1^2 + w_2^2 = 1 \implies w_1 = \sqrt{1 - w_2^2}$ and $w_2 = \sqrt{1 - w_1^2}$.
  Plugging these in yield
  \[w_1 + \rho\sqrt{1-w_1^2} + \lambda w_1 = 0 \implies w_1^2 + \rho^2(1-w_1^2) + \lambda^2 w_1^2 = 0\]
  \[w_1^2 - \rho^2 w_1^2 + \lambda^2 w_1^2 = -\rho^2\]
  \[w_1^2(1 - \rho^2 + \lambda^2) = -\rho^2\]
  \[w_1 = \pm \sqrt{\frac{-\rho^2}{1 - \rho^2 + \lambda^2}} = w_2\]
\end{example}

\end{document}

