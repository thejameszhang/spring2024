\documentclass[12pt]{scrartcl}
\usepackage[sexy]{james}
\usepackage[noend]{algpseudocode}
\setlength{\marginparwidth}{2cm}
\usepackage{answers}
\usepackage{array}
\usepackage{tikz}
\newenvironment{allintypewriter}{\ttfamily}{\par}
\usepackage{listings}
\usepackage{xcolor}
\usetikzlibrary{arrows.meta}
\usepackage{color}
\usepackage{mathtools}
\newcommand{\U}{\mathcal{U}}
\newcommand{\E}{\mathbb{E}}
\usetikzlibrary{arrows}
\Newassociation{hint}{hintitem}{all-hints}
\renewcommand{\solutionextension}{out}
\renewenvironment{hintitem}[1]{\item[\bfseries #1.]}{}
\renewcommand{\O}{\mathcal{O}}
\declaretheorem[style=thmbluebox,name={Chinese Remainder Theorem}]{CRT}
\renewcommand{\theCRT}{\Alph{CRT}}
\setlength\parindent{0pt}
\usepackage{sansmath}
\usepackage{pgfplots}

\usetikzlibrary{automata}
\usetikzlibrary{positioning}  %                 ...positioning nodes
\usetikzlibrary{arrows}       %                 ...customizing arrows
\newcommand{\eqdef}{=\vcentcolon}
\newcommand{\tr}{{\rm tr\ }}
\newcommand{\im}{{\rm Im\ }}
\newcommand{\spann}{{\rm span\ }}
\newcommand{\Col}{{\rm Col\ }}
\newcommand{\Row}{{\rm Row\ }}
\newcommand{\dint}{\displaystyle\int}
\newcommand{\dt}{\ {\rm d }t}
\newcommand{\PP}{\mathbb{P}}
\newcommand{\horizontal}{\par\noindent\rule{\textwidth}{0.4pt}}
\usepackage[top=3cm,left=3cm,right=3cm,bottom=3cm]{geometry}
\newcommand{\mref}[3][red]{\hypersetup{linkcolor=#1}\cref{#2}{#3}\hypersetup{linkcolor=blue}}%<<<changed

\tikzset{node distance=4.5cm, % Minimum distance between two nodes. Change if necessary.
         every state/.style={ % Sets the properties for each state
           semithick,
           fill=cyan!40},
         initial text={},     % No label on start arrow
         double distance=4pt, % Adjust appearance of accept states
         every edge/.style={  % Sets the properties for each transition
         draw,
           ->,>=stealth',     % Makes edges directed with bold arrowheads
           auto,
           semithick}}


% Start of document.
\newcommand{\sep}{\hspace*{.5em}}

\pgfplotsset{compat=1.18}
\begin{document}
\title{CMSC320: Introduction to Data Science}
\author{James Zhang\thanks{Email: \mailto{jzhang72@terpmail.umd.edu}}}
\date{\today}

\definecolor{dkgreen}{rgb}{0,0.6,0}
\definecolor{gray}{rgb}{0.5,0.5,0.5}
\definecolor{mauve}{rgb}{0.58,0,0.82}

\lstset{frame=tb,
  language=Java,
  aboveskip=3mm,
  belowskip=3mm,
  showstringspaces=false,
  columns=flexible,
  basicstyle={\small\ttfamily},
  numbers=left,
  numberstyle=\tiny\color{gray},
  keywordstyle=\color{blue},
  commentstyle=\color{dkgreen},
  stringstyle=\color{mauve},
  breaklines=true,
  breakatwhitespace=true,
  tabsize=3
}

\maketitle
    These are my notes for UMD's CMSC320: Introduction to Data Science, which is an
    elective. These notes are taken live in class (``live-\TeX``-ed). 
    This course is taught by Professor Fardina Fathmul. 
\tableofcontents
\newpage

\section{Big Data and Data Science Overview}

The rise of data sicnece over the last 20 years is partialkly a result of big data. 

The 3 V's of big data are 
\begin{itemize}
  \item Volume: the amount of data from myriad sources
  \item Velocity: the speed at which big data is generated
  \item Variety: the types of data (structured, semi-structured, unstructured)
\end{itemize}

In this class, we will also explore the data lifecyle which includes stages such as 
data collection, data processing, exploratory data analysis, data visualization, analysis, hypothesis testing, 
machine learing, and insight and policy discussion.

\section{Experimental Design}

\begin{definition}
  \vocab{Experimental design} is process of planning, carrying out, and analyzing experiments 
  to test a hypothesis.
  Experimental design is a crucial aspect of data science that focuses on 
  planning and conducting experiments to gather meaningful data.
\end{definition}

There are some rough experimental design steps in a research project

\begin{enumerate}
  \item Define the problem or research question the problem aims to address
  \begin{itemize}
    \item When it comes to predicting the future, choose option that maximizes your optimization criteria
  \end{itemize} 
\end{enumerate}

\begin{example}
  Online retail: you want to test whether the color of the button changes the click through rate. 
  \begin{itemize}
    \item \vocab{Problem definition}: which version of the button maximizes CTR
    \item \vocab{Optimization crtiera}: maximizng CTR
  \end{itemize}

  When it comes to setting up a dataset for this, the sample is the number of website visitors. 
  \begin{itemize}
    \item \vocab{Control group}: group with existing button color, no changes
    \item \vocab{Experimental group}: group that experiences the change that you want to test
    \item \vocab{Dependent variable}: CTR
    \item \vocab{Independent variable}: the color of the button
  \end{itemize}
\end{example}

Definitions of variable, population, sample, independent variable, dependent variable

\begin{definition}
  \vocab{Confounder variable} are extraneous variables that may affect the 
  relationships between dependent and independent variables. An additional factor 
  that is not controlled for but could potentially influence the outcomes.
\end{definition}

\begin{definition}
  \vocab{Randomization} minimize systematic confounding, reduce the risk of bias 
  in either group being enriched for confounders, and help distribute confounding 
  variables equally
\end{definition}

\begin{definition}
  \vocab{Replication} ensures certainty and suggests that confounders are less likely
  to be influencing the outcomes.
\end{definition}

Methods for collecting data
\begin{itemize}
  \item Observational

  \begin{definition}
  \vocab{Cross sectional studies}: Collects data from many different individuals 
  at one specific single point of time
\end{definition}

\begin{definition}
  \vocab{Retrospective}: look back at case studies
\end{definition}

\begin{definition}
  \vocab{Prospective}: following a cohort over a periods of time and observing changes
\end{definition}

  \item Surveys: be careful of wording to not be biased
  \item Experiments
  
\begin{definition}
  \vocab{Placebo effect}: A person believes psychologically that a certain 
  treatment is positively affecting them, even though no treatment was given at all.
\end{definition}

  \item Artificial Simulations
\end{itemize}




\end{document}

