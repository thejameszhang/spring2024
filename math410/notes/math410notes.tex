\documentclass[12pt]{scrartcl}
\usepackage[sexy]{james}
\usepackage[noend]{algpseudocode}
\setlength{\marginparwidth}{2cm}
\usepackage{answers}
\usepackage{array}
\usepackage{tikz}
\newenvironment{allintypewriter}{\ttfamily}{\par}
\usepackage{listings}
\usepackage{xcolor}
\usetikzlibrary{arrows.meta}
\usepackage{color}
\usepackage{mathtools}
\newcommand{\U}{\mathcal{U}}
\newcommand{\E}{\mathbb{E}}
\usetikzlibrary{arrows}
\Newassociation{hint}{hintitem}{all-hints}
\renewcommand{\solutionextension}{out}
\renewenvironment{hintitem}[1]{\item[\bfseries #1.]}{}
\renewcommand{\O}{\mathcal{O}}
\declaretheorem[style=thmbluebox,name={Chinese Remainder Theorem}]{CRT}
\renewcommand{\theCRT}{\Alph{CRT}}
\setlength\parindent{0pt}
\usepackage{sansmath}
\usepackage{pgfplots}

\usetikzlibrary{automata}
\usetikzlibrary{positioning}  %                 ...positioning nodes
\usetikzlibrary{arrows}       %                 ...customizing arrows
\newcommand{\eqdef}{=\vcentcolon}
\newcommand{\tr}{{\rm tr\ }}
\newcommand{\im}{{\rm Im\ }}
\newcommand{\spann}{{\rm span\ }}
\newcommand{\Col}{{\rm Col\ }}
\newcommand{\Row}{{\rm Row\ }}
\newcommand{\dint}{\displaystyle\int}
\newcommand{\dt}{\ {\rm d }t}
\newcommand{\PP}{\mathbb{P}}
\newcommand{\horizontal}{\par\noindent\rule{\textwidth}{0.4pt}}
\usepackage[top=3cm,left=3cm,right=3cm,bottom=3cm]{geometry}
\newcommand{\mref}[3][red]{\hypersetup{linkcolor=#1}\cref{#2}{#3}\hypersetup{linkcolor=blue}}%<<<changed

\tikzset{node distance=4.5cm, % Minimum distance between two nodes. Change if necessary.
         every state/.style={ % Sets the properties for each state
           semithick,
           fill=cyan!40},
         initial text={},     % No label on start arrow
         double distance=4pt, % Adjust appearance of accept states
         every edge/.style={  % Sets the properties for each transition
         draw,
           ->,>=stealth',     % Makes edges directed with bold arrowheads
           auto,
           semithick}}


% Start of document.
\newcommand{\sep}{\hspace*{.5em}}

\pgfplotsset{compat=1.18}
\begin{document}
\title{MATH410: Advanced Calculus I}
\author{James Zhang\thanks{Email: \mailto{jzhang72@terpmail.umd.edu}}}
\date{\today}

\definecolor{dkgreen}{rgb}{0,0.6,0}
\definecolor{gray}{rgb}{0.5,0.5,0.5}
\definecolor{mauve}{rgb}{0.58,0,0.82}

\lstset{frame=tb,
  language=Java,
  aboveskip=3mm,
  belowskip=3mm,
  showstringspaces=false,
  columns=flexible,
  basicstyle={\small\ttfamily},
  numbers=left,
  numberstyle=\tiny\color{gray},
  keywordstyle=\color{blue},
  commentstyle=\color{dkgreen},
  stringstyle=\color{mauve},
  breaklines=true,
  breakatwhitespace=true,
  tabsize=3
}

\maketitle
    These are my notes for UMD's MATH410: Advanced Calculus I. 
    These notes are taken live in class (``live-\TeX``-ed).
    This course is taught by Lecturer Anna Szczekutowicz. 
\tableofcontents
\newpage

\section{Set Theory Preliminaries}

This section covers the foundation of analysis, which is just the set of real numbers. 
It covers basic definitions such as $\in, \notin, \emptyset, \subseteq, =, \cap, \cup, \backslash$, so for example

\begin{definition}
  \vocab{Intersection} of $A$ and $B$ is $C= A \cap B = \{x \ | \ x \in A \text{ and } x \in B\}$
\end{definition}

Some quantifiers include $\forall, \exists, \exists!$ and some number sets include $\RR, \NN, \ZZ, \QQ, \QQ^C$. 

\begin{definition}
  The real numbers \vocab{$\RR$} satisfies 3 groups of axioms: Refer to the notes on Canvas for the Consequences of all of the following axioms.
  \begin{enumerate}
    \item Field (+, $*$) 
    \begin{itemize}
      \item Commutativity of Addition
      \item Associativity
      \item Additive Identity
      \item Additive Inverse
      \item Commutativty of Multiplcation
      \item Associativity of Multiplication
      \item Multiplicative Identity
      \item Multiplicative Inverse
      \item Distributive Property
    \end{itemize}
  The set of integers $\ZZ$ is not a field because it fails under the multiplicative inverse.
    \item Positivity
    
    There is a subset of $\RR$ denoted by $\mathcal{P}$, called the set of positive numbers for which: 
    \begin{itemize}
      \item If $x$ and $y$ are positive, then $x + y$ and $xy$ are both positive.
      \item For each $x \in \RR$, eaxctly one of the following 3 alternatives is true: $x \in \mathcal{P}$, $-x \in \mathcal{P}$, or $x = 0$
    \end{itemize}

    \item Completeness
  \end{enumerate}
\end{definition}

\begin{definition}
  \vocab{Absolute value} is defined as 
  \[|x| = \begin{cases}
    x \text{ if } x \geq 0\\
    -x \text{ if } x < 0 
  \end{cases}\]

\end{definition}

\begin{definition}
  \vocab{Triangle Inequality} is $\forall \ a, b \in \RR, |a + b| \leq |a| + |b|$
\end{definition}

\begin{proof}
  Assume without loss of generality, $a \geq b$. We will proceed with proof by cases.

  Case 1: Assume $a \geq b \geq 0$. Then $|a + b| = a + b$ by the definition of absolute value since 
  $a \geq 0, b \geq 0 \implies |a + b| = a + b = |a| + |b|$.

  Case 2: 
\end{proof}

\end{document}

