\documentclass[12pt]{scrartcl}
\usepackage[sexy]{james}
\usepackage[noend]{algpseudocode}
\setlength {\marginparwidth}{2cm}
\usepackage{answers}
\usepackage{array}
\usepackage{tikz}
\newenvironment{allintypewriter}{\ttfamily}{\par}
\usepackage{listings}
\usepackage{xcolor}
\usetikzlibrary{arrows.meta}
\usepackage{color}
\usepackage{mathtools}
\newcommand{\U}{\mathcal{U}}
\newcommand{\E}{\mathbb{E}}
\usetikzlibrary{arrows}
\Newassociation{hint}{hintitem}{all-hints}
\renewcommand{\solutionextension}{out}
\renewenvironment{hintitem}[1]{\item[\bfseries #1.]}{}
\renewcommand{\O}{\mathcal{O}}
\declaretheorem[style=thmbluebox,name={Chinese Remainder Theorem}]{CRT}
\renewcommand{\theCRT}{\Alph{CRT}}
\setlength\parindent{0pt}
\usepackage{sansmath}
\usepackage{pgfplots}

\usetikzlibrary{automata}
\usetikzlibrary{positioning}  %                 ...positioning nodes
\usetikzlibrary{arrows}       %                 ...customizing arrows
\newcommand{\eqdef}{=\vcentcolon}
\newcommand{\tr}{{\rm tr\ }}
\newcommand{\im}{{\rm Im\ }}
\newcommand{\spann}{{\rm span\ }}
\newcommand{\Col}{{\rm Col\ }}
\newcommand{\Row}{{\rm Row\ }}
\newcommand{\dint}{\displaystyle\int}
\newcommand{\dt}{\ {\rm d }t}
\newcommand{\PP}{\mathbb{P}}
\newcommand{\horizontal}{\par\noindent\rule{\textwidth}{0.4pt}}
\usepackage[top=3cm,left=3cm,right=3cm,bottom=3cm]{geometry}
\newcommand{\mref}[3][red]{\hypersetup{linkcolor=#1}\cref{#2}{#3}\hypersetup{linkcolor=blue}}%<<<changed

\tikzset{node distance=4.5cm, % Minimum distance between two nodes. Change if necessary.
         every state/.style={ % Sets the properties for each state
           semithick,
           fill=cyan!40},
         initial text={},     % No label on start arrow
         double distance=4pt, % Adjust appearance of accept states
         every edge/.style={  % Sets the properties for each transition
         draw,
           ->,>=stealth',     % Makes edges directed with bold arrowheads
           auto,
           semithick}}


% Start of document.
\newcommand{\sep}{\hspace*{.5em}}

\pgfplotsset{compat=1.18}
\begin{document}
\title{MATH410: Homework 1}
\author{James Zhang\thanks{Email: \mailto{jzhang72@terpmail.umd.edu}}}
\date{\today}

\definecolor{dkgreen}{rgb}{0,0.6,0}
\definecolor{gray}{rgb}{0.5,0.5,0.5}
\definecolor{mauve}{rgb}{0.58,0,0.82}

\lstset{frame=tb,
  language=Java,
  aboveskip=3mm,
  belowskip=3mm,
  showstringspaces=false,
  columns=flexible,
  basicstyle={\small\ttfamily},
  numbers=left,
  numberstyle=\tiny\color{gray},
  keywordstyle=\color{blue},
  commentstyle=\color{dkgreen},
  stringstyle=\color{mauve},
  breaklines=true,
  breakatwhitespace=true,
  tabsize=3
}

\maketitle

1. \begin{proof}[Solution]
  Like the hint says, rather than look considering eight
  separate cases, we will apply the Triangle Inequality twice.

  Note that 
  \[|a + b + c| = |(a + b) + c|\]
  such that $a + b \in \RR$ if $a, b \in \RR$ and this is 
  by the Positivity Axiom of the Real Numbers, $\RR$. 
  Thus, since $(a + b), c \in \RR$, we can apply the Triangle Inequality
  and obtain 
  \[|(a + b) + c| \leq |a + b| + |c|\]
  Observe the term $|a + b|$ term. Since $a, b \in \RR$, we can apply 
  the Triangle Inequality once more to get 
  \[|a + b| + |c| \leq |a| + |b| + |c|\]
  and thus we've shown that
  \[|a + b + c| \leq |a| + |b| + |c|\]
  as desired. Now for the inductive part of the prove, we wish to prove that 
  \[|a_1 + \cdots + a_n| \leq |a_1| + \cdots + |a_n| \ \forall \ n \in \NN, a_i \in \RR\] 
  Base cases: $n = 1 \implies |a_1| \leq |a_1|$ and $n = 2 \implies |a_1 + a_2| \leq |a_1| + |a_2|$ by definition of Triangle 
  Inequality.

  Inductive hypotheses: let us assume that 
  $|a_1 + \cdots + a_n| \leq |a_1| + \cdots + |a_n| \ \forall \ n \in \NN, a_i \in \RR$.

  Inductive step: Now we will show that $|a_1 + \cdots + a_n + a_{n+1}| \leq |a_1| + \cdots + |a_n| + |a_{n+1}|$.
  Starting with the left side of this inequality.
  \[|a_1 + \cdots + a_n + a_{n+1}| = |(a_1 + \cdots + a_n) + a_{n+1}|\]
  Note that $(a_1 + \cdots + a_n) \in \RR$ by the Positivity Axiom of $\RR$ and $a_{n+1} \in \RR$. Therefore, 
  we can apply the Triangle Inequality to get 
  \[|(a_1 + \cdots + a_n) + a_{n+1}| \leq |a_1 + \cdots + a_n| + |a_{n+1}|\] 
  By our Inductive step, we know that $|a_1 + \cdots + a_n| \leq |a_1| + \cdots + |a_n|$ so 
  \[|a_1 + \cdots + a_n| + |a_{n+1}| \leq |a_1| + \cdots + |a_n| + |a_{n_1}|\]
  and thus we have showed that 
  \[|a_1 + \cdots + a_n + a_{n+1}| \leq |a_1| + \cdots + |a_n| + |a_{n+1}|\]
  and this completes the proof.
\end{proof}

\newpage

2. 

\begin{proof}[Solution]

\hfill

\begin{enumerate}[a.]
  \item $\{\frac{1}{n} \ | \ n \in \NN\}$
  
An example of an upper bound of this set is $2$. An example of a lower bound
of this set is $-1$. The supremum of this set is $1$. 
The infemum of this set is $0$.

\item $\{1 - \frac{1}{3^n} \ | \ n \in \NN\}$

Example upper bound is 2. Example lower bound is -2.
Supremum is 1. Infemum is $\frac{2}{3}$ if we don't consider
0 to be in $\NN$. If it is, then the infemum is $0$. 


\item $\{\cos(\frac{n\pi}{3}) \ | \ n \in \NN\}$

An upper bound is 2. An lower bound is -2. 
The supremum is 1, and the infemum is -1. 
\end{enumerate}

\end{proof}

\newpage

3. 

\begin{proof}

Let us consider a bounded, nonempty set of real numbers $S$ such that 
$\inf S = \sup S$. On the contrary, assume $S$ contains 2 or more numbers. 
Let us denote two arbitrary elements of the set as 
$a, b \in \RR$ such that $a \neq b$, otherwise they are the same element in the set. 
Without Loss of Generality, let us say that $a < b$. By the definition of bounded, 
$\ \exists r_1, r_2$ such that $r_1 \leq a < b \leq r_2$. Therefore, 
$r_1 < r_2$. Note that the infemum and 
supremum are strict bounds on the set $S$. Let $r_1 = \inf S$ and $r_2 = \sup S$. 

By our assumption, $\inf S = \sup S \implies r_1 = r_2$, which is a contradiction since we previously 
showed that $r_1 < r_2$. Thereofre, $S$ must only contain one number. 

\end{proof}

\newpage

4a. $\underset{n\to\infty}{\lim}\frac{1}{\sqrt{n}} = 0$

Sketch: above, we want to show that $|\frac{1}{\sqrt{n}}  - 0| < \epsilon$

\[|\frac{1}{\sqrt{n}}| = \frac{1}{\sqrt{n}} < \epsilon \implies \frac{1}{\epsilon^2} < n\]
Thus, let $N = \frac{1}{\epsilon^2} < n$. 

\begin{proof}
Let $\epsilon > 0$ be given. By A.P., $\exists \ N \in \NN$ such that $\frac{1}{N}< \epsilon^2$

\[|\frac{1}{\sqrt{n}} - 0| = |\frac{1}{\sqrt{n}}| = \frac{1}{\sqrt{n}}\]
since square root of a real number is positive. 
From here, we need to relate $n$ to $N$ and then $N$ to $\epsilon$. 
Note that $n \geq N$ implies $\frac{1}{n} \leq \frac{1}{N} \implies \frac{1}{\sqrt{n}} \leq \frac{1}{\sqrt{N}}$
and $\frac{1}{N} < \epsilon^2 \implies \frac{1}{\sqrt{N}} < \epsilon$
Thus, 
\[\frac{1}{\sqrt{n}} \leq \frac{1}{\sqrt{N}} <  \epsilon\]
Therefore, we've shown that
\[|\frac{1}{\sqrt{n}} - 0| < \epsilon \ \forall \ n \geq N\implies \lim_{n\to\infty}\frac{1}{\sqrt{n}} = 0\]
\end{proof}

4b. $\underset{n\to\infty}{\lim}\frac{1}{n + 5} = 0$

Sketch: we want to show that $|\frac{1}{n + 5} - 0| < \epsilon$
\[|\frac{1}{n + 5} - 0| = |\frac{1}{n + 5}|\]
Note that the denominator will always be postive, so 
\[ = \frac{1}{n + 5} < \epsilon \implies \frac{1}{\epsilon} < n + 5 \implies \frac{1}{\epsilon} - 5 < \frac{1}{\epsilon} < n\]
Note that we can get rid of the minus 5 because $\frac{1}{\epsilon} > 0$, and 
for any number $a \in \RR^+$, $a - 5 < a$. Let us choose $N = \frac{1}{\epsilon} < n$. 

\begin{proof}
  Let $\epsilon > 0$ be given. By A.P., $\exists N \in \NN$ such that $\frac{1}{N} < \epsilon \implies $

  \[|\frac{1}{n + 5} - 0| = |\frac{1}{n + 5}| = \frac{1}{n + 5} < \frac{1}{n}\]
  Recall that $n \geq N$ and so $\frac{1}{n} \leq \frac{1}{N}$
  \[\frac{1}{n} \leq \frac{1}{N} < \epsilon\]
  Therefore, we've shown that 
  \[|\frac{1}{n + 5} - 0| < \epsilon \ \forall \ n \geq N \implies \lim_{n\to\infty}\frac{1}{n+5} = 0\]
  as desired.
\end{proof}

\newpage

5a.  Sketch: From calculus, we know the limit is $1$, but we will prove it rigorously. 
We want to show that $|\frac{n^2}{n^2 + n} - 1| < \epsilon$. 
\[|\frac{n^2}{n^2 + n} - 1| = |\frac{n^2}{n^2 + n} - \frac{n^2+ n}{n^2 + n}| = |-\frac{n}{n^2 + n}|\]
Both the numerator and denomintor will always be positive, so
\[|-\frac{n}{n^2 + n}| = \frac{n}{n^2 + n} < \frac{n}{n^2} = \frac{1}{n} < \epsilon \implies \frac{1}{\epsilon} < n\]
Thus let us choose $N = \frac{1}{\epsilon}$.

\begin{proof}

Let $\epsilon > 0$ be given. By A.P., $\exists \ N \in \NN$ such that $\frac{1}{N} < \epsilon$
which implies that 
\[|\frac{n^2}{n^2 + n} - 1| = |-\frac{n}{n^2 + n}| = \frac{n}{n^2 + n} < \frac{n}{n^2} = \frac{1}{n}\]
Look at the above sketch for more detail and for additional logic. Now, recall that 
$n \geq N$ which implies that 
\[\frac{1}{n} \leq \frac{1}{N} < \epsilon\]
Therefore,
\[|\frac{n^2}{n^2 + n} - 1| < \epsilon \ \forall \ n \geq \frac{1}{\epsilon} \implies \lim_{n\to\infty}\frac{n^2}{n^2 + n} = 1\]
as desired.

\end{proof}

5b. Sketch: We want to show that $|\frac{\sin n}{n} - 0| < \epsilon$. 
\[|\frac{\sin n}{n} - 0| = |\frac{\sin n}{n}| \leq |\frac{1}{n}| = \frac{1}{n} < \epsilon \implies\]
Choose $N = \frac{1}{\epsilon}$

\begin{proof}
  
  Let $\epsilon > 0$ be given. By A.P., $\exists \ N \in \NN$ such that $\frac{1}{N} < \epsilon$ and so 
  \[|\frac{\sin n}{n} - 0| = |\frac{\sin n}{n}| \leq |\frac{1}{n}|\]
  since $|\sin n| \leq 1 \ \forall \ n$.
  \[|\frac{1}{n}| = \frac{1}{n}\]
  Recall that $n \geq N \implies \frac{1}{n} \leq \frac{1}{N}$.
  \[\frac{1}{n} \leq \frac{1}{N} < \epsilon\]
  by our choice of $N$. Therefore, we have shown that given some $\epsilon > 0$, we can find an 
  $\frac{1}{N} < \epsilon$ such that $\ \forall \ n \geq N$
  \[|\frac{\sin n}{n } - 0| < \epsilon \ \forall \ n \geq N \implies \lim_{n\to\infty}\frac{\sin n}{n} = 0\] 

\end{proof}

\newpage

6. 

\begin{proof}

\hfill

We are given that $\underset{n\to\infty}{\lim} a_n = \underset{n\to\infty}{\lim} b_n = s$. 
By the definition of convergence,
\[\forall \ \epsilon > 0 \ \exists \ N_1 \in \NN \text{ s.t. } \forall \ n \geq N_1, |a_n - s| < \epsilon\]
Therefore, 
\[|a_n - s| < \epsilon \implies -\epsilon < a_n - s < \epsilon\]
$\forall \ n \geq N_1$.
Similarly, for the same $\epsilon > 0$, we can say that 
\[-\epsilon < b_n - s < \epsilon\]
$\forall \ n \geq N_2$. Furthermore, since $a_n \leq s_n \leq b_n \ \forall \ n$, we can subtract $s$ from all terms such that 
we obtain 
\[a_n - s\leq s_n - s\leq b_n - s \ \forall \ n \geq N\]
Choose $N = \max(N_1, N_2)$. Therefore, we can say that 
\[-\epsilon < a_n - s \leq s_n - s \leq b_n - s < \epsilon\]
\[-\epsilon < s_n - s < \epsilon\]
\[|s_n - s| < \epsilon\]
for all $n \geq N$, and the last step is by definition of absolute value. Now, by the definition of convergence, 
since for any $\epsilon > 0$, there exists some $N = \max(N_1, N_2) \in \NN$ such that 
$\ \forall \ n \geq N$,
\[|s_n - s| < \epsilon \implies \lim_{n\to\infty}a_n = s\]
and this completes the proof.

\end{proof}

\newpage

7. 

\begin{proof}

\hfill

Note that this is an "if and only if" statement, so we must prove both directions.
  
$\Longrightarrow$ Suppose we are given that $\{c_n\}$ converges to $c$. By the definition of convergence, given $\epsilon > 0$
\[|c_n - c| < \epsilon\]
for all $n \geq N, N \in \NN$. Note that we can expand the expression in the absolute value to be 
\[|c_n - c - 0| = |(c_n - c) - 0| < \epsilon\]
which satisfies the structure $|a_n - L| < \epsilon$, where here $a_n = c_n - c$ and $L = 0$. 
Therefore, $\underset{n\to\infty}{\lim}c_n - c = 0$. Note that the above is a strict equality, 
but we can also use the Comparison Lemma since 
given our $\epsilon > 0$ and our choice of $N$, 
\[|(c_n - c) - 0| \leq 1|c_n - c| \ \forall \ n \geq N\]
Since there exists some $a = 1, a \in \RR^+$, then we conclude that $\{c_n - c\}$ converges to $0$.

\hfill

$\Longleftarrow$ Supose we are given instead that $c_n - c$ converges to $0$. By the definition of convergnce, 
given some $\epsilon > 0$, we write that 
\[|(c_n - c) - 0| < \epsilon\]
for all $n \geq N, N \in \NN$. Simplifying the expression in absolute value, we get 
\[|(c_n - c) - 0| = |c_n - c| < \epsilon\]
which again satisfies the definition of convergence where $a_n = c_n$ and $L = c$. Once more, 
we could have used the Comparison Lemma since 
\[|c_n - c| \leq 1|(c_n - c) - 0| \ \forall \ n \geq N\]
since $\ \exists \ 1 \in \RR^+$, then we conclude that $\{c_n\}$ converges to $c$. 

\hfill

Thus, we have proven that the sequence $\{c_n\}$ converges to $c$ iff the sequence $\{c_n - c\}$
converges to $0$, as desired.

\end{proof}

\end{document}

